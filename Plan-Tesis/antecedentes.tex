\chapter{Antecedentes}
\renewcommand{\figurename}{Figura}

\section{Estado del Arte}
%En el artículo titulado "Phthalate-induced testosterone/androgen receptor pathway disorder on spermatogenesis and antagonism of lycopene" publicado en el año de 2022, se detalla que la vía de señalización de testosterona (T)/receptor de andrógenos (AR) está involucrada en el mantenimiento de la espermatogénesis y la fertilidad masculina. Los resultados obtenidos en la investigación demostraron que el ftalato de mono-2-etilhexilo (MEHP) causó daño mitocondrial y daño oxidativo, por lo que se determinó que esta sustancia química seria amenaza para el progreso de la espermatogénesis. Sin embargo, en la investigación también realizaron estudios antagonistas del licopeno frente a la alteración que produce los ftalatos en el trastorno de la vía del receptor de andrógenos/ testosterona, teniendo como resultados que este suplemento LYC es una agente antioxidante natural que inhibe los cambios producidos por los ftalatos frente la función espermatogénica de los testículos. En general, este estudio revelo un papel fundamental para la transducción de señales T/AR en la fertilidad masculina y proporciono información prometedora sobre el papel protector de LYC en los trastornos reproductivos masculinos inducidos por ftalatos.\cite{b12}


