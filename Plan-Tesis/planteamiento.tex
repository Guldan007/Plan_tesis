\chapter{Planteamiento de la Investigación}
\renewcommand{\figurename}{Figura}
La Diabetes se encuentra entre las diez enfermedades con alta mortalidad en adultos, presentando una prevalencia del 9,3\% en el año 2019 y según predicciones se puede esperar un aumento del 0,9\% para el año 2030. %Rachin%
Esta alarma no es indiferente a la búsqueda de soluciones alternativas para tratar mencionada enfermedad, existen diversos métodos y tratamientos que se han expuesto a la sociedad como son medicamentos asociados a reducir la glucosa en la sangre como los son la metformina que es uno de más usados actualmente.%Buscar idea%
Las investigaciones nos indican que existen otros blancos disponibles como lo es el tratamiento de la DMT2 aumentando la prevalencia de la activación de la via de señalización dependiente de GLP1, evitando su inhibición por la enzima Dipeptidil peptidaza 4 (DPP4) mencionada proteína puede inhibirse por diversos fármacos del grupo de las gliptinas, como lo es la Vildagliptina entre otras. %Buscaridea%

Se ha querido encontrar diversos metabolitos que puedan evitar la interacción farmacéutica o solo reducirla para poder ser reemplazada por algún metabolito proveniente de agentes naturales como lo son las plantas. %buscar idea%
En esta investigación nos centraremos en el potencial de diversos metabolitos encontrados en el aloe vera para poder ser comparados y encontrar diferencias significativas entre el uso de los agentes farmacológicos y los agentes naturales. viendo las diferencias podremos afirmar si los agentes derivados del aloe vera podrían ser similares, o mejores que los agentes farmacológicos. %buscar idea% 

\section{Problemática de la investigación}

Según lo encontrado por Rachi et al podemos afirmar que la prevalencia de Diabetes irá en aumento hasta llegar a un nivel de aumento del 0,9-1\% de prevalencia en aproximadamente 5 años, es por ello que se requiere de nuevas tecnologías que puedan ayudar al tratamiento o control de mencionada patología.
Dentro de la problemática principal como lo es el aumento de la prevalencia de los casos de diabetes a nivel global, esto trae consigo problemas asociados como lo son las neuropatías, daño renal, retinopatías entre otras.%Publications%
Por otro lado se ha observado que la Metformina ha empezado a presentar algunos efectos secundarios en gente adulta como lo son malestares gástricos e intestinales los cuales han forzado a los tratamientos ayudarse del grupo de los fármacos como las gliptinas para poder ayudar a estos cambios siendo la solución la inhibición de la proteína DPP4.

Finalmente el consumo de fármacos seguirá en crecimiento durante este aumento de la prevalencia de Diabetes es por ello que es mejor optar por otras opciones aparte de los fármacos y tratamientos convencionales enfocándonos en la naturaleza, es por ello que haciendo un análisis de posibles inhibidores postulamos al aloe vera como uno de los posibles portadores de metabolitos secundarios que puedan contener compuestos que interaccionen a nivel del intestino y estómago con fundamento a los antecedentes que presenta mencionada planta frente a la diabetes.

\section{Pregunta de investigación}
Dentro del análisis de datos obtenidos de los antecedentes del aloe vera con la diabetes ¿Es posible mencionar que los metabolitos secundarios encontrados en el aloe vera presenten mayor capacidad inhibitoria que los  fármacos convencionales que interactúan con la proteína DPP4 para el control de la Diabetes Mellitus tipo 2 (DMT2)?

\section{Justificación}
%social 

%tecnológica
Se ha visto que diversos métodos aplicados para el tratamiento de la diabetes mellitus tipo 2, y debido a la prevalencia nacional encontrada que nos habla de 7\% de prevalencia en Perú. %
%Resultados La prevalencia nacional estimada de diabetes fue del 7,0\% (IC del95\% 5,3\% a 8,7\%) y fue de 8,4\% (IC del 95\% 5,6\% a 11,3\%) en Lima metropolitana. 
%Ventajas de la biología computacional: Explicar las ventajas del enfoque computacional para estudiar la interacción enzima-inhibidor. Mencionar técnicas específicas que se utilizarán (ej: docking molecular, dinámica molecular, análisis de componentes principales) y cómo estas técnicas ayudarán a superar las limitaciones de los métodos experimentales. Detallar las bases de datos o software que se utilizarán.

%Hipótesis y objetivos: Formular una hipótesis específica y clara sobre la interacción y definir los objetivos de la tesis. Estos objetivos deben estar directamente relacionados con el hueco de conocimiento identificado. Los objetivos deben ser medibles y alcanzables en el marco de una tesis de pregrado.
%Originalidad (a nivel de pregrado): Aunque puede que no se trate de una investigación completamente novedosa a nivel mundial, se debe justificar la originalidad del trabajo a nivel de pregrado. Esto podría implicar:
 %   El análisis de una nueva variante de la enzima o inhibidor.
 %   La aplicación de una nueva metodología computacional a un sistema previamente estudiado.
  %  Un enfoque novedoso en el análisis de los datos obtenidos.
  %  Una combinación de diferentes técnicas computacionales para un análisis más completo.
%DOI
%https://doi.org/10.1016/j.compbiolchem.2024.108145
  %Base de datos
%https://www.ebi.ac.uk/chembl/explore/drug_mechanisms/STATE_ID:wFt0kVdHXqmQoTue0brSuQ%3D%3D

%ambieltal


\section{Alcance}


\section{Objetivos}

\subsection{General}


\subsection{Específicos}

\begin{itemize}
 \item 
\item 
\item 

\end{itemize}

\section{Hipótesis}



\section{Variables e indicadores}

 \begin{table}[htbp]
 \caption{Cuadro de variables}\label{t1}
\begin{tabular}{lllccc}
\hline
\multicolumn{3}{c}{\textbf{Variables}}                       & \textbf{Variable}      & \textbf{Indicadores}                 & \textbf{Unidades} \\ \hline
\multicolumn{3}{c}{\multirow{2}{*}{\textbf{Independientes}}} &   &   & \\ \cline{4-6} 
\multicolumn{3}{c}{}                                         &   &   & \\ \hline
\multicolumn{3}{l}{\multirow{4}{*}{\textbf{Dependientes}}}   &   &   & \\ \cline{4-6} 
\multicolumn{3}{l}{}                                         &   &   & \\ \cline{4-6} 
\multicolumn{3}{l}{}                                         &   &   & \\ \cline{4-6} 
\multicolumn{3}{l}{}                                         &   &   & \\ \hline
\end{tabular}
\end{table}

\section{Tipo y Nivel de Investigación}



