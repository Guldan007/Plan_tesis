\chapter{Marco Teórico}

 \section {Diabetes y desafíos actuales}
 Actualmente la Diabetes Mellitus Tipo 2 (DM2) es una de las mas grandes preocupaciones epidemiológicas globales. Según estadísticas revisadas en el campo de la salud pública revelan una progresión alarmante de esta enfermedad metabólica que trasciende las fronteras geográficas y socioeconómicas. En el contexto latinoamericano, investigaciones %Guevara Tirado (2)%
 demuestran la creciente prevalencia de la DM2, particularmente en poblaciones con características demográficas y genéticas específicas, especialmente en sectores con pocos recursos de salud.
 
 La complejidad de esta enfermedad radica no solo en su impacto directo sobre la salud individual, sino en sus profundas implicaciones socioeconómicas. El incremento sostenido de casos de DM2 representa una carga significativa para los sistemas de salud, generando costos directos e indirectos que afectan tanto a individuos como a comunidades enteras.

 Siendo la Diabetes Mellitus Tipo 2 (DMT2) una de las afecciones más comunes en el campo de la salud pública por los resultados encontrados en diversos estudios es prioridad de la comunidad científica el encontrar soluciones y alternativas que puedan acortar o controlar la evolución de tan preocupante epidemia silenciosa como lo nombran algunos autores. 

 \section {Fundamentos Moleculares y Fisiopatológicos de la Dipeptidil-Peptidasa IV}
 A nivel molecular, la Diabetes Mellitus Tipo 2 se configura como un fenómeno multifactorial caracterizado por alteraciones complejas en los mecanismos de señalización metabólica. La proteína dipeptidil peptidasa-4 (DPP4) emerge como un elemento central en esta intrincada red de procesos bioquímicos, debido a la forma en la que esta se relaciona con la recepción de glucosa celular y mecanismos que nos ayudan a poder controlar el nivel de glucosa en la sangre. Esta proteína presenta una afinidad por moléculas peptídicas como los son las incretinas que ayudan al metabolismo de ATP en las células.

 Algunas investigaciones como las realizadas por Kathiraven, han profundizado en la comprensión de los mecanismos por los cuales los inhibidores de DPP-IV intervienen no solo en el control glucémico, sino también en las potenciales manifestaciones cardiovasculares asociadas a la diabetes.%Kathiraven et al. (3)%
 Esto debido a que la DPP4 actúa como una enzima proteolítica crucial que degrada hormonas incretinas fundamentales como el péptido similar al glucagón-1 (GLP-1) y el polipéptido insulinotrópico dependiente de glucosa (GIP), que estas a su vez están relacionadas a la fisiología principal de la generación de energía celular como controladores de la admisión de glucosa y la regulación de canales iónicos.
 
 Esta catálisis o digestion enzimática genera la activación de las proteínas transportadoras de glucosa, mencionadas proteínas tienen el objetivo de habilitar el mecanismo de generación de ATP, es decir que si esta digestion existe por la disminución de las incretinas estaríamos teniendo mayor concentración de glucosa en el medio, es por ello la importancia de la inhibición de la proteína DPP4 para que las incretinas de los pacientes que presentan DMT2 mantengan funcionando la linea metabólica de generación de ATP. En otras palabras, esta degradación enzimática impacta directamente en la capacidad del organismo para mantener una homeostasis glucémica equilibrada, alterando la secreción de insulina, la sensibilidad de los tejidos y los mecanismos de regulación metabólica.

 \section {Estrategias Farmacológicas Contemporáneas}
 La evolución de las estrategias terapéuticas para el manejo de la DM2 ha experimentado transformaciones significativas en las últimas décadas. %Shao et al. (4)%
 Se han documentado exhaustivamente la síntesis de moléculas inhibidoras de DPP-4, destacando los avances en el desarrollo de compuestos cada vez más selectivos y con perfiles de seguridad optimizados, aparte de ser enfocados como un tratamiento complementario a tratamientos convencionales como por ejemplo a la metformina, que un fármaco muy usado para el control del avance de DMT2.
 
 Han realizado contribuciones fundamentales en la comprensión de las relaciones estructura-actividad de estos inhibidores. Sus estudios revelan la complejidad inherente al diseño de moléculas que cumplan simultáneamente criterios de eficacia, selectividad y tolerancia.%Singhal et al. (5) 
 Este análisis será controlado por los perfiles de Absorción, Distribución,
 Metabolismo, Excreción y Toxicidad (ADMET) los cuales son normas que validan el uso de los ligandos e interacciones con las proteínas evaluando su farmacocinética y farmacodinámica. 
 
 \section {El Aloe Vera como potencial recurso natural}
 Diversos análisis toman al Aloe vera como una fuente prometedora de compuestos bioactivos con potencial terapéutico. Las múltiples perspectivas emergentes en la biosíntesis y bioactividades de los metabolitos secundarios de esta planta, ampliando el horizonte de investigación más allá de los enfoques farmacológicos tradicionales. %Ushasree et al. (6)

 Según algunos estudios en especial el desarrollado por Ushasree et al. encontraron propiedades antivirales, inmunomodulatorias, antiinflamatorias, antidiabéticas y metabolitos capaces de inhibir enzimas. Uno de los focos de investigación son metabolitos como la Aloina, Aloesin entre otros.%Ushasree et al. (6)
 Por ejemplo, existen contribuciones pioneras al identificar un derivado dipirrólico del Aloe vera con capacidades inhibitorias documentadas sobre la enzima DPP-IV, validando experimentalmente el potencial de los extractos naturales en el abordaje terapéutico de la diabetes.%Prasannaraja et al. (7)

 \section{Tecnologías computacionales para simulaciones moleculares}
 El campo de las simulaciones moleculares con recursos computacionales como son los softwares de dinámicas moleculares, programas de cálculos cuánticos entre otros ayudan a reducir el error de generar un resultado innecesario cuando se desea encontrar interacciones entre proteínas y ligandos o simular entornos de interacción entre proteínas, esto ayuda a acercarnos más a un resultado que ayudaría a reducir los costos de diversos materiales acortando el gasto económico entre otros aspectos.
 
 Como se expuso en el párrafo anterior, las tecnologías computacionales actuales representan actualmente una revolución metodológica en la investigación farmacológica. Algunos investigadores presentaron enfoques innovadores que integran aprendizaje automático y simulaciones de dinámica molecular para el descubrimiento de compuestos con potencial terapéutico.%De La Torre et al. (8) 
 Otros desarrollaron estrategias de cribado virtual que permiten analizar sistemáticamente derivados de flavonoides naturales, reduciendo significativamente los tiempos y costos asociados con las metodologías experimentales tradicionales. %Lu et al. (9)
 Como puede observarse en los antecedentes existen muchas ventajas que nos ofrecen las diferentes estrategias computacionales

 \subsection{Modelado Molecular Avanzado}
 
 Los estudios computacionales contemporáneos ofrecen herramientas de una precisión sin precedentes. Por ejemplo, Suleiman et al. Han demostrado la capacidad de las simulaciones moleculares para predecir interacciones proteína-ligando con niveles de detalle que superan ampliamente las aproximaciones experimentales convencionales. %Suleiman et al. (10)
 Esto podría complementarse con estudios realizados por Maya y Yadav que han expandido estas metodologías incorporando perfilados ADMET que permiten evaluar no solo la potencial actividad inhibitoria, sino también parámetros fundamentales como biodisponibilidad, toxicidad y estabilidad metabólica.% Maya y Yadav (11)
 Finalmente podemos afirmar con toda seguridad que las ventajas que nos ofrecen las simulaciones computaciones son mas que una opción una ventaja que nos ayudará en la reducción de costos y reducción de errores experimentales.

 \section{Tendencias de Investigación Emergentes}
 Existen diversas aplicaciones novedosas como lo son los análisis detallados de derivados o las pruebas de screenings para encontrar metabolitos que presentan capacidad inhibitoria en la enzima DPP4. Como las encontradas por Masyita;que nos muestra que los agentes moleculares derivados de las moléculas como la emodina que se encuentra en las raíces del aloe vera, que presentan modificaciones sutiles, pueden incrementar significativamente la potencia inhibitoria sobre la DPP4. %Masyita et al. (12)
 Sin embargo otros autores amplían el espectro de exploración incorporando extractos de fuentes vegetales diversas, evidenciando que la naturaleza continúa siendo una fuente inexplorada de potenciales agentes terapéuticos.%Zollapi et al. (13)

 Luego de lo declarado podemos afirmar que las fuentes de metabolitos como las plantas presentan varias aplicaciones que aun no han sido desarrolladas completamente, es por ello que en esta investigación se plantea el objetivo de centrarnos en la mayoría de metabolitos encontrados en el aloe vera con aplicaciones clínicas para luego ser comparadas con los agentes farmacéuticos comúnmente recetados para el tratamiento de la DMT2 como lo son los fármacos de la familia de las -gliptinas, cuyo mecanismo es el de inhibir la proteína DPP4 y servir como tratamiento complementario a otros mayormente usados por el sector salud.   

 \subsection{Desarrollo de objetivos}
 La presente investigación se fundamenta en la necesidad de desarrollar alternativas terapéuticas que combinen eficacia, seguridad, accesibilidad y origen natural.
 La integración de metabolitos de Aloe vera con técnicas avanzadas de biología computacional representa una aproximación innovadora y multidisciplinaria para abordar los desafíos del tratamiento de la Diabetes Mellitus Tipo 2.
 En estas aplicaciones se presentan diversos tipos de experimentos que presentan resultados prometedores generando el fundamento que el Aloe Vera presenta metabolitos con aplicación en inhibición de enzimas como lo es en este caso, la DPP4.

 Existe mucha información que afirma que la DPP4 es un blanco util en el tratamiento de la DMT2 y existen medicamentos de primera línea que interactúan con ella para producir su inhibición, pero asegurar un futuro libre de acceso farmacéutico usando derivados naturales como lo es el aloe vera podría reducir la demanda de fármacos y generar mayor investigación sobre plantas que presenten metabolitos similares que puedan ayudar a las diversas sociedades a generar una independencia de fármacos sintéticos y complementar los tratamientos comunes con diversos extractos naturales reduciendo la carga química que presenta normalmente un paciente con DMT2.


 