\chapter{Planteamiento de la Investigación}
\renewcommand{\figurename}{Figura} 

\section{Problemática de la investigación}

Dentro de la problemática principal como lo es el aumento de la prevalencia de los casos de diabetes a nivel global, esto trae consigo problemas asociados como lo son las neuropatías, daño renal, retinopatías entre otras.%Publications%
Por otro lado se ha observado que la Metformina ha empezado a presentar algunos efectos secundarios en gente adulta como lo son malestares gástricos e intestinales los cuales han forzado a los tratamientos ayudarse del grupo de los fármacos como las gliptinas para poder ayudar a estos cambios siendo la solución la inhibición de la proteína DPP4.

Finalmente el consumo de fármacos seguirá en crecimiento durante este aumento de la prevalencia de Diabetes es por ello que es mejor optar por otras opciones aparte de los fármacos y tratamientos convencionales enfocándonos en la naturaleza, es por ello que haciendo un análisis de posibles inhibidores postulamos al aloe vera como uno de los posibles portadores de metabolitos secundarios que puedan contener compuestos que interaccionen a nivel del intestino y estómago con fundamento a los antecedentes que presenta mencionada planta frente a la diabetes.

\section{Pregunta de investigación}
Dentro del análisis de datos obtenidos de los antecedentes del aloe vera con la diabetes ¿Es posible mencionar que los metabolitos secundarios encontrados en el aloe vera presenten mayor capacidad inhibitoria que los  fármacos convencionales que interactúan con la proteína DPP4 para el control de la Diabetes Mellitus tipo 2 (DMT2)?

\section{Justificación}
%social 
%En el ámbito social, presentamos multiples falencias como es la evolución lenta de implementación de nuevas técnicas de organización nacional sanitaria, esto se demuestra en las estadísticas encontradas sobre la prevalencia de diabetes en el Perú cuyo estado predice un aumento de prevalencia de la enfermedad en estos años, es por ello que una de las soluciones es emplear nuevas tecnologías para que esta y muchas de las nuevas tecnologías podría brindarnos alternativas para poder emplear metabolitos naturales de plantas nativas del Perú para así brindar un respaldo de prevención y control de la enfermedad frente a la poca distribución de fármacos o pacientes cuyo tratamiento diario con medicamentos sintéticos generando asi posibles resistencias en el organismo.  


%tecnológica

%Se ha visto que diversos métodos aplicados para el tratamiento de la diabetes mellitus tipo 2, y debido a la prevalencia nacional encontrada que nos habla de 7\% de prevalencia en Perú. Tomamos en cuenta que, las nuevas tecnologías son la esperanza de evitar grandes gastos en el control de enfermedades y síntesis de fármacos, en este caso las tecnologías de predicción de interacciones y resultados pre-experimentales nos ayudan a dilucidar que diferentes metabolitos naturales podrían tener un efecto igual o superior a los fármacos comúnmente comercializados para el fin de la inhibición de la enzima DPP4. 
%Por otra parte los resultados obtenidos nos abrirán paso a nuevas aplicaciones de plantas importantes empleadas actualmente como medicina alternativa al consumo excesivo de medicamentos sintéticos, finalmente al emplear de nuevas tecnologías computacionales en el tratamiento de la diabetes mellitus como lo es la bioinformática nos podría ayudar a encontrar respuesta no solo a nuevos metabolitos, sino también las barreras de aplicación de tratamientos a personas cuya genética rechaza tratamientos convencionales.%

%Resultados La prevalencia nacional estimada de diabetes fue del 7,0\% (IC del95\% 5,3\% a 8,7\%) y fue de 8,4\% (IC del 95\% 5,6\% a 11,3\%) en Lima metropolitana. 
%Ventajas de la biología computacional: Explicar las ventajas del enfoque computacional para estudiar la interacción enzima-inhibidor. Mencionar técnicas específicas que se utilizarán (ej: docking molecular, dinámica molecular, análisis de componentes principales) y cómo estas técnicas ayudarán a superar las limitaciones de los métodos experimentales. Detallar las bases de datos o software que se utilizarán.

%Hipótesis y objetivos: Formular una hipótesis específica y clara sobre la interacción y definir los objetivos de la tesis. Estos objetivos deben estar directamente relacionados con el hueco de conocimiento identificado. Los objetivos deben ser medibles y alcanzables en el marco de una tesis de pregrado.
%Originalidad (a nivel de pregrado): Aunque puede que no se trate de una investigación completamente novedosa a nivel mundial, se debe justificar la originalidad del trabajo a nivel de pregrado. Esto podría implicar:
 %   El análisis de una nueva variante de la enzima o inhibidor.
 %   La aplicación de una nueva metodología computacional a un sistema previamente estudiado.
  %  Un enfoque novedoso en el análisis de los datos obtenidos.
  %  Una combinación de diferentes técnicas computacionales para un análisis más completo.
%DOI
%https://doi.org/10.1016/j.compbiolchem.2024.108145
  %Base de datos
%https://www.ebi.ac.uk/chembl/explore/drug_mechanisms/STATE_ID:wFt0kVdHXqmQoTue0brSuQ%3D%3D

%ambieltal
%En el Perú se encuentra una de las mayores base multidiversificadas de flora y fauna de Sudamérica. En esta ocasión se explica algunas de las ventajas de una especie de las múltiples variedades de flora que existe en el país, el aloe vera; demostrando la riqueza que podemos encontrar en la química extraída de forma natural en las plantas. Es por ello, que considero, que reportes que nos ayuden a generar conciencia de cuidar la diversificación de la flora son necesarios debido a las aplicaciones que podemos encontrar en ellas, sin necesidad del uso de gasto económico en recursos para síntesis de medicamentos y reduciendo considerablemente contaminantes de suelo y agua a partir de medicamentos que no son metalizados completa o parcialmente por nuestro organismo y que en muchas ocasiones llegan a ser desechados  sin tomar importancia del gran impacto que estos pueden tener frente a animales y plantas.

%Es por ello que este proyecto tiene como uno de los objetivos el argumentar en la importancia del cuidado de la flora resaltando la importancia que estas plantas podrían tener y ayudando a reducir el consumo excesivo de medicamentos, optimizando los tratamientos convencionales y concientizar acerca del cuidado ala naturaleza.
En el ámbito social, presentamos múltiples falencias en implementación de nuevas técnicas para el desarrollo de tratamientos, esto se demuestra en las estadísticas encontradas sobre la prevalencia de diabetes en el Perú cuyo estado predice un aumento de prevalencia de la enfermedad en estos años, es por ello que una de las soluciones es emplear nuevas tecnologías para que esta y muchas de las nuevos métodos brinden alternativas para poder emplear metabolitos naturales de plantas nativas del Perú, para así dar un respaldo de prevención y control de la enfermedad, que actualmente está en vías de crecimiento.
Se ha visto que diversos métodos aplicados para el tratamiento de la diabetes mellitus tipo 2; pero notamos que, las nuevas tecnologías son la esperanza de evitar grandes gastos en el control de enfermedades y síntesis de fármacos, es por ello que, los resultados obtenidos nos abrirán paso a nuevas aplicaciones de plantas importantes empleadas actualmente como medicina ,alternativa al consumo excesivo de medicamentos sintéticos, finalmente al emplear de nuevas tecnologías computacionales en el tratamiento de la diabetes mellitus como lo es la bioinformática nos podría ayudar a encontrar respuesta a las las barreras de aplicación de tratamientos a personas cuya genética rechaza tratamientos convencionales.
Es por ello, que considero, que reportes que nos ayuden a generar conciencia de cuidar la diversificación de la flora son necesarios debido a las aplicaciones que podemos encontrar en ellas, sin necesidad del uso de gasto económico en recursos para síntesis de medicamentos y reduciendo considerablemente contaminantes de suelo y agua.

\section{Alcance}
La presente investigación se desarrollará bajo un alcance descriptivo-correlacional, con el propósito de analizar comparativamente los metabolitos del aloe vera y los fármacos convencionales en relación con su poder inhibitorio de la proteína DPP4.
El componente descriptivo permitirá caracterizar detalladamente las propiedades moleculares de los metabolitos del aloe vera, documentando sus características estructurales y comportamiento bioquímico. Simultáneamente, el enfoque correlacional facilitará el establecimiento de relaciones sistemáticas entre los diferentes compuestos, permitiendo cuantificar y comparar su capacidad inhibitoria.
Se contempla un análisis exhaustivo que incluirá la identificación de metabolitos activos, la evaluación de su potencial farmacológico y la determinación de correlaciones estadísticamente significativas entre los diferentes compuestos estudiados. El objetivo central radica en comprender la potencial eficacia del aloe vera como alternativa terapéutica en la modulación de la enzima DPP4.

\section{Objetivos}

\subsection{General}
Proponer la interacción de metabolitos secundarios del Aloe Vera como inhibidores potenciales de la dipeptidil peptidasa-4 (DPP4) postulándolos como alternativas a tratamiento de Diabetes Mellitus tipo 2 (DMT2)

\subsection{Específicos}

\begin{itemize}
 \item Identificar las características metabólicas encontradas en la hidrólisis de la proteína GLP1 por la enzima DPP4 en sus estados basales
 \item Analizar y generar una base de datos de metabolitos de alto interés clínico proveniente del Aloe Vera al igual que de fármacos de alta importancia en el tratamiento de la diabetes mellitus tipo 2 
 \item Cuantificar y comparar la capacidad inhibitoria de los metabolitos del aloe vera con fármacos convencionales utilizando softwares de biología computacional, determinando y evidenciando los mecanismos de interacción molecular.

\end{itemize}

\section{Hipótesis}
Los metabolitos encontrados en el aloe vera presentarán propiedades inhibitorias superiores a los fármacos comerciales usados en la primera línea para el tratamiento de Diabetes Mellitus Tipo 2 (DMT2).  


\section{Variables e indicadores}

 \begin{table}[htbp]
 \caption{Cuadro de variables}\label{t1}
  \begin{tabular}{llcp{3.5cm}p{3.5cm}p{3.5cm}cp{3.5cm}}
  \hline
  \multicolumn{3}{c}{\textbf{Variables}}                       & \textbf{Variable}      & \textbf{Indicadores}                 & \textbf{Unidades} \\ \hline
  \multicolumn{3}{c}{\multirow{2}{*}{\textbf{Independientes}}} 
  & Ligandos & Propiedades fisicoquímicas & IC50,ADMET\\ \cline{4-6} 
  \multicolumn{3}{c}{}                                         &Proteínas  & Condiciones experimentales  & pH, Temperatura \\ \hline
  \multicolumn{3}{l}{\multirow{3}{*}{\textbf{Dependientes}}}   &Binding Energy   &Energía libre de Gibbs, Energía de interacción   & KJ/mol \\ \cline{4-6} 
  \multicolumn{3}{l}{}                                         &Proximidad Molecular   &Distancia entre átomos   &Angstroms \\ \cline{4-6} 
  \multicolumn{3}{l}{}                                         &Dinámica Molecular   &RMSD y RMSF   &nm/ns \\ \hline 
  \end{tabular}
\end{table}
%Independientes
%Metabolitos del aloe vera (identificados y seleccionados)
%Fármacos convencionales de referencia (identificados y seleccionados)
%Características fisicoquímicas de los ligandos

%Dependientes
%Valores de afinidad de unión (docking molecular)
%Porcentaje de inhibición calculado computacionalmente
%Cambios conformacionales de la proteína DPP4
\section{Tipo y Nivel de Investigación}
El tipo de investigación usada para este análisis es básica siendo una investigación fundamental ya que nos da a conocer una alternativa que no tiene aplicación inmediata pero si unos resultados validados en el campo cuántico químico y fisicoquímicas, generando resultados para el avance de tecnologías que podrían ayudar a la optimización de tratamientos contra la DMT2  


