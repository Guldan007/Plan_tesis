\chapter{Antecedentes}
\renewcommand{\figurename}{Figura}

\section{Estado del Arte}
La Diabetes Mellitus Tipo 2 (DMT2) representa uno de los mayores desafíos de salud pública contemporáneos \cite{GuevaraTirado2024}. Su prevalencia global ha ido en aumento, generando una necesidad urgente de estrategias terapéuticas innovadoras y alternativas \cite{Rachin2024}. Los inhibidores de la dipeptidil peptidasa-4 (DPP-4) han emergido como una línea terapéutica prometedora para el manejo de esta condición metabólica.Algunos libros como lo es publicado por "Exon Publications" donde se afirma que, la diabetes tipo 2 es una afección crónica caracterizada por la resistencia a la insulina y alteración de la secreción de insulina, lo que conduce a niveles elevados de glucosa en sangre. Esta publicación nos brinda información sobre la Diabetes Tipo 2. Sus explicaciones abarcan los tipos, factores de riesgo, epidemiología, causas, síntomas, fisiopatología, complicaciones, diagnóstico, tratamiento y pronóstico de Diabetes Tipo 2. Dentro de este libro nos muestran algunas de las afecciones añadidas a la diabetes como lo es la falla renal, neuropatías, retinoptologías y afecciones estomacales como ulceras, entre otros\cite{Publications2024}.

Los inhibidores de DPP-4 han ganado reconocimiento significativo en el control glucémico de pacientes con DMT2 \cite{Masyita2024}. Estas moléculas actúan modulando la actividad enzimática de la dipeptidil peptidasa-4, lo que resulta en una mejora del metabolismo de la glucosa \cite{Singhal2024,Shao2024}.Estudios recientes han destacado la importancia de explorar nuevas fuentes de inhibidores de DPP-4, especialmente aquellos derivados de origen natural \cite{Maya2024, DeLaTorre2024}. En este contexto, los metabolitos secundarios de plantas medicinales emergen como candidatos promisorios.

Dentro de el libro antes mencionado nos brindan las explicaciones del porque la diabetes es una afección de alta preocupación entre la sociedad, y es necesario encontrar soluciones inmediatas. En el Perú, también se encuentra una alta prevalencia como lo menciona, Alberto Guevara Tirado, autor del artículo titulado: Determinación del riesgo de diabetes mellitus tipo 2 en la población peruana. Encuesta Demográfica y de Salud Familiar 2022. Él afirma que la diabetes mellitus tipo 2 (DM2) es una enfermedad cuya causa implica factores endógenos y exógenos, representando un problema de salud pública mundial. 
En este trabajo observacional, analítico, retrospectivo y transversal. Analizó 29.000 adultos no diabéticos cuyos datos provinieron de la Encuesta Demográfica y de Salud Familiar (ENDES 2022) para analizar el riesgo de DM2 en la población peruana. Obteniendo que existen alteraciones del indice de masa corporal como de presión arterial en adultos con alto riesgo de DMT2, siendo mas común en hombres que en mujeres siendo un total del 49\% de hombres y un 22.30\% de mujeres, el autor también nos comenta que existe minima ayuda social mientras que sería beneficioso hacer estudios estadísticos generales para la detección de grupos de riesgo a nivel nacional constantemente.\cite{GuevaraTirado2024}

Aloe vera ha sido reconocida tradicionalmente por sus propiedades medicinales \cite{Ushasree2024}. Algunos estudios previos demostraron el efecto antihiperglucémico de Aloe vera pero con resultados inconsistentes.Pero Indah et al en el artículo "The effect of Aloe vera on fasting blood glucose levels in pre-diabetes and type 2 diabetes mellitus: A systematic review and meta-analysis" tuvo como objetivo asumir cuantitativamente el efecto de Aloe vera en el ayuno de glucosa en sangre en la pre-diabetes y diabetes mellitus tipo 2 mediante un metaanálisis.\cite{Budiastutik2022}
Se usaron como buscadores principales para analizar ensayos clínicos que pueden haber evaluado el aloe vera en el ayuno de glucosa en sangre publicado entre 2011 y 2021: PubMed, Scopus, Springer Link, Science Direct, Proquest, y Google Scholar. Se utilizó Aloe vera como la única intervención haciendo que el efecto agrupado de Aloe vera en la glucosa en sangre en ayunas se evaluó usando el modelo de efecto aleatorio, y el sesgo de publicación fue evaluado por las parcelas de Funnel y Fail Safe-N. En mencionado paper se tomaron los siguientes resultados, mostrando que el Aloe vera redujo significativamente la glucosa en sangre en ayunas (-0,35 [IC 95\%, -1454, -0,616] mg/dL; p-0,001) en comparación con el control. 

El aloe vera puede tener un efecto más notable en los hombres, IMC no superior a 30 mg/kg2, diabetes mellitus tipo 2, administrado para una dosis superior a más de 8 semanas, dosis a 200 mg y administración de cápsulas. Sin embargo, se encontró una alta heterogeneidad en los estudios. Los que nos sugiere la productividad del Aloe Vera en el tratamiento de la diabetes mellitus tipo 2 $(DMT2)$.\cite{Ushasree2024,Budiastutik2022}

Investigaciones específicas han identificado componentes moleculares de Aloe vera con potencial inhibidor de DPP-4. Prasannaraja et al. \cite{Prasannaraja2020} documentaron un derivado dipirrólico que inhibe in vitro la enzima DPP-IV, proporcionando evidencia preliminary de su mecanismo de acción.

Existen algunos estudios experimentales que encontraron gran potencial en el aloe vera como tratamiento de la DMT2 como el titulado: Comparative Metabolic Properties of Aloe Vera Extracts and Sitagliptin in Diabetic Rats. En este artículo se compararon los efectos hipoglucemientes del aloe vera frente a los efectos de la Sitagliptina en ratas con diabetes. viendo que ambos tratamientos redujeron significativamente los niveles de glucosa en la sangre en ayunas, concluyendo en que el aloe vera presenta una gran aplicación frente a la DMT2.\cite{Javaid2022}

Aunque actualmente la investigación farmacéutica ha integrado estrategias computacionales avanzadas para el descubrimiento de compuestos inhibidores de DPP-4 \cite{DeLaTorre2024, Lu2023}. Estas metodologías incluyen: Cribado virtual,Modelado molecular
Análisis de relación estructura-actividad (SAR) y Predicciones de ADMET, que nos ayudan a manejar gran cantidad de datos por significancia estadística y relevancia clínica.Estudios recientes han comenzado a explorar sistemáticamente los metabolitos de Aloe vera utilizando herramientas computacionales \cite{Zollapi2023, Suleiman2024}. Estas investigaciones buscan identificar y caracterizar moléculas con potencial inhibidor de DPP-4.La investigación de Kazeem et al. \cite{Kazeem2021} destaca la importancia de los alimentos funcionales con potencial inhibidor de DPP-4 en el manejo de la DMT2, subrayando el papel de los metabolitos vegetales en las estrategias terapéuticas contemporáneas.Por otro lado, Kathiraven et al. \cite{Kathiraven2024} exploraron la relación entre inhibidores de DPP-IV en el tratamiento de la diabetes y las enfermedades cardiovasculares, ampliando la comprensión de sus potenciales beneficios terapéuticos.

A pesar de los avances prometedores, persisten desafíos significativos como la estandarización de extractos de Aloe vera
elucidación completa de mecanismos moleculares,desarrollo de preparaciones farmacéuticas optimizadas y estudios clínicos de mayor escala y robustez. Es por ello que podemos concluir afirmando que la investigación emergente sugiere que los metabolitos secundarios de Aloe vera representan una línea de investigación prometedora para el desarrollo de nuevos inhibidores de DPP-4 \cite{Singhal2024, Shao2024}. Sin embargo, se requieren estudios adicionales para validar completamente su potencial terapéutico.


