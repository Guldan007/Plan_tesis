\chapter{Antecedentes}
\renewcommand{\figurename}{Figura}

\section{Estado del Arte}
Algunos libros como lo es publicado por "Exon Publications" donde se afirma que, la diabetes tipo 2 es una afección crónica caracterizada por la resistencia a la insulina y alteración de la secreción de insulina, lo que conduce a niveles elevados de glucosa en sangre. Esta publicación nos brinda información sobre la Diabetes Tipo 2. Sus explicaciones abarcan los tipos, factores de riesgo, epidemiología, causas, síntomas, fisiopatología, complicaciones, diagnóstico, tratamiento y pronóstico de Diabetes Tipo 2. 
Dentro de este libro nos muestran algunas de las afecciones añadidas a la diabetes como lo es la falla renal, neuropatías, retinoptologías y afecciones estomacales como ulceras, entre otros.\cite{Publications2024}

Dentro de el libro antes mencionado nos brindan las explicaciones del porque la diabetes es una afección de alta preocupación entre la sociedad, y es necesario encontrar soluciones inmediatas. En el Perú, también se encuentra una alta prevalencia como lo menciona, Alberto Guevara Tirado, autor del artículo titulado: Determinación del riesgo de diabetes mellitus tipo 2 en la población peruana. Encuesta Demográfica y de Salud Familiar 2022. Él afirma que la diabetes mellitus tipo 2 (DM2) es una enfermedad cuya causa implica factores endógenos y exógenos, representando un problema de salud pública mundial. 
En este trabajo observacional, analítico, retrospectivo y transversal. Analizó 29.000 adultos no diabéticos cuyos datos provinieron de la Encuesta Demográfica y de Salud Familiar (ENDES 2022) para analizar el riesgo de DM2 en la población peruana. Obteniendo que existen alteracions del indice de masa corpoal como de presion arterial en adultos con alto riesfo de DMT2, siendo mas comun en hombres que en mujeres siendo un total del 49\% de hombres y un 22.30\% de mujeres, el autor tambien nos comenta que existe posco sector en vias de ayuda mientras que sería beneficioso hacer estudios estadistios generales para la detección de grupos de riesgo a nivel nacional constantemente.\cite{GuevaraTirado2024}

El Aloe vera, por otro lado, es una planta medicinal propia de la gran diversidad biológica. Algunos estudios previos demostraron el efecto antihiperglucémico de Aloe vera pero con resultados inconsistentes. 
Pero Indah et al en el artìculo "The effect of Aloe vera on fasting blood glucose levels in pre-diabetes and type 2 diabetes mellitus: A systematic review and meta-analysis" tuvo como objetivo asumir cuantitativamente el efecto de Aloe vera en el ayuno de glucosa en sangre en la pre-diabetes y diabetes mellitus tipo 2 mediante un metaanálisis.
Se usaron como buscadores principales para analizar ensayos clìnicos que pueden haber evaluado el aloe vera en el ayuno de glucosa en sangre publicado entre 2011 y 2021: PubMed, Scopus, Springer Link, Science Direct, Proquest, y Google Scholar. Se utilizó Aloe vera como la única intervención haciendo que el efecto agrupado de Aloe vera en la glucosa en sangre en ayunas se evaluó usando el modelo de efecto aleatorio, y el sesgo de publicación fue evaluado por las parcelas de Funnel y Fail Safe-N. En mencionado paper se tomaron los siguirntes resultados, mostrando que el Aloe vera redujo significativamente la glucosa en sangre en ayunas (-0,35 [IC 95\%, -1454, -0,616] mg/dL; p-0,001) en comparación con el control. 

El aloe vera puede tener un efecto más notable en los hombres, IMC no superior a 30 mg/kg2, diabetes mellitus tipo 2, administrado para una dosis superior a más de 8 semanas, dosis a 200 mg y administración de cápsulas. Sin embargo, se encontró una alta heterogeneidad en los estudios. Los que nos sugiere la productividad del Aloe Vera en el tratamiento de la diabetes mellitus tipo 2 $(DMT2)$.\cite{Budiastutik2022}

Existen algunos estudios experimentales que encontraron gran potencial en el aloe vera como tratamiento de la DMT2 como el titulado: Comparative Metabolic Properties of Aloe Vera Extracts and Sitagliptin in Diabetic Rats. En este artículo se compararon los efectos hipoglucemientes del aloe vera frente a los efectos de la Sitagliptina en ratas con diabetes. viendo que ambos tratamientos redujeron significativamente los niveles de glucosa en la sangre en ayunas, concluyendo en que el aloe vera presenta una gran aplicación frente a la DMT2. \cite{Budiastutik2022,JavaidandZafar}





%En el artículo titulado "Phthalate-induced testosterone/androgen receptor pathway disorder on spermatogenesis and antagonism of lycopene" publicado en el año de 2022, se detalla que la vía de señalización de testosterona (T)/receptor de andrógenos (AR) está involucrada en el mantenimiento de la espermatogénesis y la fertilidad masculina. Los resultados obtenidos en la investigación demostraron que el ftalato de mono-2-etilhexilo (MEHP) causó daño mitocondrial y daño oxidativo, por lo que se determinó que esta sustancia química seria amenaza para el progreso de la espermatogénesis. Sin embargo, en la investigación también realizaron estudios antagonistas del licopeno frente a la alteración que produce los ftalatos en el trastorno de la vía del receptor de andrógenos/ testosterona, teniendo como resultados que este suplemento LYC es una agente antioxidante natural que inhibe los cambios producidos por los ftalatos frente la función espermatogénica de los testículos. En general, este estudio revelo un papel fundamental para la transducción de señales T/AR en la fertilidad masculina y proporciono información prometedora sobre el papel protector de LYC en los trastornos reproductivos masculinos inducidos por ftalatos.\cite{b12}


